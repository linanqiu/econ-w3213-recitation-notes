\documentclass[11pt]{scrartcl}
\usepackage{dominatrix}
\usepackage{tabularx}
\newcolumntype{Y}{>{\centering\arraybackslash}X}
\usepackage{colortbl}
\usepackage{pgfplots}
\newcommand{\jon}{J\'{o}n }
\newcommand{\oneth}{\ensuremath{\frac{1}{3}}}
\newcommand{\twoth}{\ensuremath{\frac{2}{3}}}
\newcommand{\ve}{\varepsilon}
\pgfplotsset{compat=1.9}
\definecolor{light-gray}{gray}{0.75}
\title{Great Depression}
\subject{ECON W3213 Spring 2014 \jon Steinsson}
\author{Linan Qiu, lq2137}
\begin{document}

\maketitle

\begin{abstract}
This set of recitation notes covers the \textbf{Great Depression}. This is in no way a substitute for attending lectures, but just in case you dozed off or checked your boyfriend's Facebook page while \jon was working Calculus magic on the board, this set of notes may save you.

If you were listening in class, you'd know that this topic goes far more than just these 4 factors. In fact, \jon goes very far into the history behind the Great Depression. I didn't cover those here because you can catch up on your own with the readings.
\end{abstract}

So far, we started with a medieval money model and we added gold, gold coverage, and bank deposits (the bank multiplier) into the model. Before we add on more stuff, let's take a look at what happened historically during the great depression to understand the historical origins of those innovations. History time!

Let's answer the question of "What caused the Great Depression?" Even till today, there's no strict consensus. In fact, 

\begin{quote}
Economics is the only field in which two people can share a Nobel Prize for saying opposing things\footnote{Myrdal and Hayek shared a Nobel prize for saying basically opposite things}.
\end{quote}

However, Bernanke (he's kind of a big shot) identified four factors

\begin{enumerate}
\item Excessively Tight Monetary Policy in 1928 to 1929
\item Sterling Crisis
\item Easing in Spring of 1932
\item Run on Dollar
\end{enumerate}

We shall look at the second in particular detail because we can analyze them using very simple game theory. This is probably one of the only set of notes that have little to no math, so rejoice if you will. Pfft.

Side note, if you don't know what the Great Depression is, wiki\footnote{\url{http://en.wikipedia.org/wiki/Great_depression}} it. And no, the Great Depression is not your finals week.


\end{document}